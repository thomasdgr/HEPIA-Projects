% USAGE : https://github.com/tomncooper/pandoc-gls

% Insérez les termes pour le glossaire ici.
% Ne mettez pas de points à la fin d'une entrée, ils sont mis pour vous !
\newglossaryentry{scraper_g}{name=scraper,description=Logiciel permettant la récupération en masse de documents}
\newglossaryentry{scraping_g}{name=scraping,description=Utilisation d'un \gls{scraper_g}}
\newglossaryentry{framework_g}{name=framework,description=Un framework est un ensemble de composants permettant de poser les fondations d'un logiciel de façon à ne pas tout programmer de zéro}
\newglossaryentry{captcha_g}{name=captcha,description=Un système permettant de différencier une machine d'un humain en demandant par exemple de faire une addition ou lire du texte}

% Insérez les termes pour la table des acronymes ici.
% Ne mettez pas de points à la fin d'une entrée, ils sont mis pour vous !
\newacronym{OPM_a}{OPM}{Orestis Pileas Malaspinas}
\newacronym{TPG_a}{TPG}{Transports Publics Genevois}
\newacronym{GTFS_a}{GTFS}{General Transit Feed Specification}
\newacronym{OSM_a}{OSM}{Open Street Map}
\newacronym{CFF_a}{CFF}{Chemins de fer fédéraux suisses}
\newacronym{SITG_a}{SITG}{Système d'Information du Territoire à Genève}
\newacronym{HEPIA_a}{HEPIA}{Haute École du Paysage, d'Ingénierie et d'Architecture de Genève}
\newacronym{KML_a}{KML}{Keyhole Markup Language}